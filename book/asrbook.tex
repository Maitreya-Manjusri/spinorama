\documentclass[11pt,oneside,a4paper]{report}

\usepackage[margin=1cm]{geometry}
\usepackage{graphicx}
\usepackage{caption}
\graphicspath{ {./tmp/} }

\begin{document}
\title{ASR Speaker reviews}
\author{P. Aubert \& A. Majidimehr}
\date{2020}
\maketitle

\listoffigures

\pagebreak

\chapter{Introduction}

AudioSciencereview \emph{aka} ASR \footnote{https://www.audiosciencereview.com} is a website created by A. Majidimehr.
His aim is to provide objective data for audio equipements.

This booklet provides an introduction and a set of measurements for speakers measured on ASR.

\section{Why it is important?}

Audio world is plaggued with subjective reviews. They do not help you to make a decision. What ones like is
not necesary what you like. This reviews usually gets more dithyrambic with the price of the speaker and
get you confused.

Science has been studing speakers for a long time and there is a vast amount of scientific research available.
If you only want to read one book on this topic, we advise you to read \cite{Toole2013}. It proves over and over
that you should trust science over subjective reviews, which is what you usually do when you go to a medical
doctor or when you get your car checked.

In a nutshell, a few graphs characterize pretty well a speaker. You get unbiased informations and then it is
easy to compare between speakers. Some vendors do provides this informations in manuals but very few do. Why?
Possibly because there is very little correlation between price and quality.

\section{What are the most important parameters for a speaker?}

\begin{itemize}
\item A flat frequency response:

  you expect a speaker to reproduce the sound as it has been recorded without emphasis.
\item Bass extension:

  bass reproduction is important but requires bigger speakers to get enough of them. Small speakers
  usually need a subwoofer to cover the full spectrum.
\item Constant directivity:

  when you move off axis (you are not exactly in front of a speaker), you expect the sound
  to vary regulary).
\end{itemize}

This parameters are now standardized. You can get the standard focument for free at ...


\chapter{Measurements}

\section{How to read this measurement?}

\pagebreak

% for speaker in speakers:
\begin{center}
   \includegraphics[width=0.6\textwidth]{${speakers[speaker]['image']}}
   \captionof{figure}{${speakers[speaker]['title']}}
 \end{center}
% endfor

\bibliography{audio}{}
\bibliographystyle{plain}

\end{document}
