\documentclass{tufte-book}

\IfFileExists{bergamo.sty}{\usepackage[osf]{bergamo}}{}% Bembo
\IfFileExists{chantill.sty}{\usepackage{chantill}}{}% Gill Sans

\usepackage{geometry}
\usepackage{graphicx}
\setkeys{Gin}{width=\linewidth,totalheight=\textheight,keepaspectratio}
\graphicspath{ {./tmp/} }
\usepackage{booktabs}

\begin{document}
\title{ASR Speaker reviews}
\author{P. Aubert \& A. Majidimehr}
\date{2020}
\maketitle

% \tableofcontents
% \listoffigures

\pagebreak

\chapter{Introduction}

AudioSciencereview \emph{aka} ASR \footnote{https://www.audiosciencereview.com} is a website created by A. Majidimehr.
His aim is to provide objective data for audio equipements.

This booklet provides an introduction and a set of measurements for speakers measured on ASR.
The introduction explains first why it is important to have measurements and then how to interpret them.
You should be able to decide by yourself how good a loudspeaker is for your use case and if the price
seems adapted.

\section{Why it is important?}

Audio world is plaggued with subjective reviews. They do not help you to make a decision. What ones like is
not necesary what you like. This reviews usually gets more dithyrambic with the price of the speaker and
get you confused.

Science has been studing speakers for a long time and there is a vast amount of scientific research available.
If you only want to read one book on this topic, we advise you to read \cite{Toole2013}. It proves over and over
that you should trust science over subjective reviews, which is what you usually do when you go to a medical
doctor or when you get your car checked. We do not discuss form factor, style or design: they are important criteria
but they are subjective by nature.

In a nutshell, a few graphs characterize pretty well a speaker. You get unbiased informations and then it is
easy to compare between speakers. Some vendors do provides this informations in manuals but very few do. Why?
Possibly because there is very little correlation between price and quality.

\section{What are the most important parameters for a speaker?}

Pragmatically, price, size and look\&feel usually come first and depend on usage. Another question is
active or passive? Active means amplifier and dac are built into the speaker (like most bluetooth speakers)
and passive means a box with drivers, you use a separate amplifier and dac, that's the classical case.

After that, for a given price, what are the speakers which are the most likely to have a pleasing sound? For you
but also for most people as we will see.

Here are 3 key parameters:
\begin{itemize}
\item A flat frequency response:

  you expect a speaker to reproduce the sound as it has been put on a CD, a vinyl or a streaming service
  without emphasis. Flat means in a anechoic room, this should not be the case in your room where you expect
  a downward curve, the so-called X-curve.
\item Bass extension:

  bass reproduction is important but requires bigger speakers to go low and to have enough volume. Small speakers
  usually need a subwoofer to cover the full spectrum.
\item Constant directivity:

  when you move off axis (you are not exactly in front of a speaker, or right in the middle), you expect the sound
  to vary regulary).
\end{itemize}

This parameters are now standardized. You can get the standard document for free \cite{CEA2034}.

\chapter{How to understand a Spinorama?}

\section{Where do the measurements come from?}
There are 3 sources of data:
\begin{enumerate}
\item AudioScienceReview: Amir use a special measurement machine built by Klipel. This machine can emulate
  an anechoid room and provides very precise measurements in a normal room. Methodology is backed by a
  theoritical framework \cite{Klippel2016} and multiple comparisons with anechoid rooms.
\item 3D3A a laboratory within Princeton university. They did a directivity analysis of many speakers \cite{Tylka2015}
  and published related papers \cite{Tylka2015}, \cite{Tylka2014}.
\item Data provided by vendors. They may or not be accurate. Independant measurements by other organisations
  help to qualify and improve confidence in this measurements.
\end{enumerate}

\section{How to read this measurements?}

\chapter{Two case studies}

\section{Genelec 8341A \emph{v.s.} Harbeth 30.0}

This section compare two speakers which are both well known and expensive:
\begin{enumerate}
\item Genelec 8341A is a coaxial speakers, a 3 ways, an active studio monitor design for near-field listening.
\item Harbeth Monitor 30.0 is a larger 2 ways, passive monitor.
\end{enumerate}

\begin{figure*}[ht]
  \includegraphics[width=0.45\linewidth]{tmp/Genelec-8341A/CEA2034}
  \includegraphics[width=0.45\linewidth]{tmp/Harbeth-Monitor-30-0/CEA2034}
\end{figure*}
This Genelec speaker is extremely flat on axis while having an flat directivity.
This Harbeth is less flat on axis and has bump into its directivity.

Looking at the reflections, we see the same pattern:
\begin{figure*}[ht]
  \includegraphics[width=0.45\linewidth]{tmp/Genelec-8341A/Early-Reflections}
  \includegraphics[width=0.45\linewidth]{tmp/Harbeth-Monitor-30-0/Early-Reflections}
\end{figure*}
The Harbeth will benefit from a thick carpet and ideally some room treatment on
the ceiling in order to minimise the bounces.

\begin{figure*}[ht]
  \includegraphics[width=0.45\linewidth]{tmp/Genelec-8341A/Estimated-In-Room-Response}
  \includegraphics[width=0.45\linewidth]{tmp/Harbeth-Monitor-30-0/Estimated-In-Room-Response}
\end{figure*}
Both speakers behave very well with a downward trending curve.

\begin{figure*}[ht]
  \includegraphics[width=0.45\linewidth]{tmp/Genelec-8341A/SPL-Horizontal-Contour}
  \includegraphics[width=0.45\linewidth]{tmp/Harbeth-Monitor-30-0/SPL-Horizontal-Contour}
  \\\vspace{\baselineskip}
  \includegraphics[width=0.45\linewidth]{tmp/Genelec-8341A/SPL-Horizontal-Contour}
  \includegraphics[width=0.45\linewidth]{tmp/Harbeth-Monitor-30-0/SPL-Horizontal-Contour}
\end{figure*}
Here we see textbook result from the coaxial. The passive speaker has issue around its crossover.
We can have a look at the same data in a radar plot or in a flat graph:

\begin{figure*}[ht]
  \includegraphics[width=0.45\linewidth]{tmp/Genelec-8341A/SPL-Horizontal-Radar}
  \includegraphics[width=0.45\linewidth]{tmp/Harbeth-Monitor-30-0/SPL-Horizontal-Radar}
  \\\vspace{\baselineskip}
  \includegraphics[width=0.45\linewidth]{tmp/Genelec-8341A/SPL-Horizontal-Radar}
  \includegraphics[width=0.45\linewidth]{tmp/Harbeth-Monitor-30-0/SPL-Horizontal-Radar}
\end{figure*}

\begin{figure*}[ht]
  \includegraphics[width=0.45\linewidth]{tmp/Genelec-8341A/SPL-Horizontal}
  \includegraphics[width=0.45\linewidth]{tmp/Harbeth-Monitor-30-0/SPL-Horizontal}
  \\\vspace{\baselineskip}
  \includegraphics[width=0.45\linewidth]{tmp/Genelec-8341A/SPL-Horizontal}
  \includegraphics[width=0.45\linewidth]{tmp/Harbeth-Monitor-30-0/SPL-Horizontal}
\end{figure*}
The contour, the radar and the graph represent the same values. Sometimes it is easier
to read one or the other.


\section{Infinity v.s. Verdant Audio Bambusa G 1}

This section compare 2 other speakers: both are passive bookshelves:
\begin{enumerate}
\item Infinity R162, 270\$ a pair.
\item Verdant Audio Bambusa MG 1, 5000\$ a pair.
\end{enumerate}

\begin{figure*}[ht]
  \includegraphics[width=0.45\linewidth]{tmp/Infinity-R162/CEA2034}
  \includegraphics[width=0.45\linewidth]{tmp/Verdant-Audio-Bambusa-MG-1/CEA2034}
\end{figure*}

Looking at the reflections, we see the same pattern:
\begin{figure*}[ht]
  \includegraphics[width=0.45\linewidth]{tmp/Infinity-R162/Early-Reflections}
  \includegraphics[width=0.45\linewidth]{tmp/Verdant-Audio-Bambusa-MG-1/Early-Reflections}
\end{figure*}

\begin{figure*}[ht]
  \includegraphics[width=0.45\linewidth]{tmp/Infinity-R162/Estimated-In-Room-Response}
  \includegraphics[width=0.45\linewidth]{tmp/Verdant-Audio-Bambusa-MG-1/Estimated-In-Room-Response}
\end{figure*}

\begin{figure*}[ht]
  \includegraphics[width=0.45\linewidth]{tmp/Infinity-R162/SPL-Horizontal-Contour}
  \includegraphics[width=0.45\linewidth]{tmp/Verdant-Audio-Bambusa-MG-1/SPL-Horizontal-Contour}
  \\\vspace{\baselineskip}
  \includegraphics[width=0.45\linewidth]{tmp/Infinity-R162/SPL-Horizontal-Contour}
  \includegraphics[width=0.45\linewidth]{tmp/Verdant-Audio-Bambusa-MG-1/SPL-Horizontal-Contour}
\end{figure*}

\begin{figure*}[ht]
  \includegraphics[width=0.45\linewidth]{tmp/Infinity-R162/SPL-Horizontal-Radar}
  \includegraphics[width=0.45\linewidth]{tmp/Verdant-Audio-Bambusa-MG-1/SPL-Horizontal-Radar}
  \\\vspace{\baselineskip}
  \includegraphics[width=0.45\linewidth]{tmp/Infinity-R162/SPL-Horizontal-Radar}
  \includegraphics[width=0.45\linewidth]{tmp/Verdant-Audio-Bambusa-MG-1/SPL-Horizontal-Radar}
\end{figure*}

\begin{figure*}[ht]
  \includegraphics[width=0.45\linewidth]{tmp/Infinity-R162/SPL-Horizontal}
  \includegraphics[width=0.45\linewidth]{tmp/Verdant-Audio-Bambusa-MG-1/SPL-Horizontal}
  \\\vspace{\baselineskip}
  \includegraphics[width=0.45\linewidth]{tmp/Infinity-R162/SPL-Horizontal}
  \includegraphics[width=0.45\linewidth]{tmp/Verdant-Audio-Bambusa-MG-1/SPL-Horizontal}
\end{figure*}

Conclusion: this two speakers have very close graphs and very close scores. They are likely
to sound very close to each other.

\chapter{Measurements}

% for speaker in speakers:
\begin{figure*}[p]
  \includegraphics[width=\linewidth]{${speakers[speaker]['image']}}
  \caption{${speakers[speaker]['title']}}
\end{figure*}
% endfor

\chapter{Tables}

\begin{marginfigure}
  \includegraphics[width=\linewidth]{stats/distribution.png}
  \caption{Scores distribution}
\end{marginfigure}

\begin{figure*}[p]
  \includegraphics[width=\linewidth]{stats/spread.png}
  \caption{Scores}
\end{figure*}

\begin{figure*}[p]
\fbox{\includegraphics[width=0.45\linewidth]{stats/lfx_hz.png}}
\hfill
\fbox{\includegraphics[width=0.45\linewidth]{stats/nbd_on.png}}
\\\vspace{\baselineskip}
\fbox{\includegraphics[width=0.45\linewidth]{stats/nbd_pir.png}}
\hfill
\fbox{\includegraphics[width=0.45\linewidth]{stats/sm_pir.png}}
\end{figure*}

\pagebreak

\begin{table}[ht]
\begin{center}
\begin{tabular}{r|rrrr|rrrrr}
Speaker & \multicolumn{4}{r|}{Standard Measurements} & \multicolumn{5}{r}{Preference Scores from Harmann} \\\

\hline

 & -3dB & -6dB & Dev & LFX  & LFQ & NBD & NBD & SM  & Pref \\\
 & (Hz) & (Hz) &(dB) & (Hz) &     &  ON & PIR & PIR & Score \\\

\hline

% for speaker_name in meta:
% if 'pref_rating' in meta[speaker_name]:
${speaker_name}
&
% if 'estimates' in meta[speaker_name] and meta[speaker_name]['estimates'][1] != -1:
${meta[speaker_name]['estimates'][1]}
% endif
&
% if 'estimates' in meta[speaker_name] and meta[speaker_name]['estimates'][2] != -1:
${meta[speaker_name]['estimates'][2]}
% endif
&
% if 'estimates' in meta[speaker_name] and meta[speaker_name]['estimates'][3] != -1:
${meta[speaker_name]['estimates'][3]}
% endif
&
% if 'pref_rating' in meta[speaker_name]:
%   if 'lfx_hz' in meta[speaker_name]['pref_rating']:
${meta[speaker_name]['pref_rating']['lfx_hz']}
%   endif
% endif
&
% if 'pref_rating' in meta[speaker_name]:
%   if 'lfq' in meta[speaker_name]['pref_rating']:
${meta[speaker_name]['pref_rating']['lfq']}
%   endif
% endif
&
% if 'pref_rating' in meta[speaker_name]:
${meta[speaker_name]['pref_rating']['nbd_on_axis']}
% endif
&
% if 'pref_rating' in meta[speaker_name]:
${meta[speaker_name]['pref_rating']['nbd_pred_in_room']}
% endif
&
% if 'pref_rating' in meta[speaker_name]:
${meta[speaker_name]['pref_rating']['sm_pred_in_room']}
% endif
&
% if 'pref_rating' in meta[speaker_name]:
%   if 'pref_score' in meta[speaker_name]['pref_rating']:
\emph{${meta[speaker_name]['pref_rating']['pref_score']}}
%   endif
% endif
\\\

% endif
% endfor
\end{tabular}
\end{center}
\caption{Speakers key metrics}
\end{table}


\bibliography{audio}{}
\bibliographystyle{plain}

\end{document}
